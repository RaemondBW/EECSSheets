\documentclass[3pt,landscape]{article}
%ss[10pt,landscape]{article}
\usepackage{multicol}
\usepackage{calc}
\usepackage{ifthen}
\usepackage[landscape]{geometry}
\usepackage{amsmath,amsthm,amsfonts,amssymb}
\usepackage{color,graphicx,overpic}
\usepackage{hyperref}


\pdfinfo{
/Title (final.pdf)
/Creator (TeX)
/Producer (pdfTeX 1.40.0)
/Author (Raemond Bergstrom-Wood)
/Subject (Algorithms)
/Keywords (pdflatex, latex,pdftex,tex)}

% This sets page margins to .5 inch if using letter paper, and to 1cm
% if using A4 paper. (This probably isn't strictly necessary.)
% If using another size paper, use default 1cm margins.
\ifthenelse{\lengthtest { \paperwidth = 11in}}
    { \geometry{top=.5in,left=.5in,right=.5in,bottom=.5in} }
    {\ifthenelse{ \lengthtest{ \paperwidth = 297mm}}
        {\geometry{top=1cm,left=1cm,right=1cm,bottom=1cm} }
        {\geometry{top=1cm,left=1cm,right=1cm,bottom=1cm} }
    }

% Turn off header and footer
\pagestyle{empty}

% Redefine section commands to use less space
\makeatletter
\renewcommand{\section}{\@startsection{section}{1}{0mm}%
                            {-1ex plus -.5ex minus -.2ex}%
                            {0.5ex plus .2ex}%x
                            {\normalfont\large\bfseries}}
\renewcommand{\subsection}{\@startsection{subsection}{2}{0mm}%
                            {-1explus -.5ex minus -.2ex}%
                            {0.5ex plus .2ex}%
                            {\normalfont\normalsize\bfseries}}
\renewcommand{\subsubsection}{\@startsection{subsubsection}{3}{0mm}%
                            {-1ex plus -.5ex minus -.2ex}%
                            {1ex plus .2ex}%
                            {\normalfont\small\bfseries}}
\makeatother

% Define BibTeX command
\def\BibTeX{{\rm B\kern-.05em{\sc i\kern-.025em b}\kern-.08em
    T\kern-.1667em\lower.7ex\hbox{E}\kern-.125emX}}

% Don't print section numbers
\setcounter{secnumdepth}{0}


\setlength{\parindent}{0pt}
\setlength{\parskip}{0pt plus 0.5ex}

%My Environments
\newtheorem{example}[section]{Example}
% -----------------------------------------------------------------------

\def\ci{\perp\!\!\!\perp}

\begin{document}
\raggedright
\footnotesize
\begin{multicols}{3}


% multicol parameters
% These lengths are set only within the two main columns
%\setlength{\columnseprule}{0.25pt}
\setlength{\premulticols}{1pt}
\setlength{\postmulticols}{1pt}
\setlength{\multicolsep}{1pt}
\setlength{\columnsep}{2pt}

\begin{center}
    \Large{\underline{CS 170 Final Cheat Sheet}} \\
\end{center}

\subsection*{Euclid's GCD: \(O(n^{3})\)}
\begin{verbatim}
def gcd(a,b):
    if b==0:
        return a
    return gcd(b, a mod b)
\end{verbatim}

\subsection*{Extended GCD: \(O(n^{3})\)}
\begin{verbatim}
def extended-gcd(a,b):
    if b==0:
        return (1, 0, a)
    (x', y', d) = extended-gcd(b, a mod b)
    return (y', x' - floor(a/b)*y', d)
\end{verbatim}
if d divides a and b and \(d=ax+by\) for some integers s and y,\\
then d = gcd(a, b)

\subsection*{Multiplicative Inverse}
inverse of a,
\[ax \equiv 1 (\texttt{mod } N)\]
for any \(a (\texttt{mod } N)\), a has a multiplicative inverse if and only if they are relatively prime, gcd(a,N) = 1

\subsection*{Fermat's Little Theorem}
todo add the proof of this\\
given a prime (or carmichael) p,
\[a^{p-1} \equiv 1 (\texttt{mod } p)\]

\subsection*{RSA Euler's Theorem}
\[m^{(p-a)(q-1)}=1 (\texttt{mod } p)\]

\subsection*{Master's Theorem}
If
\[T(n)=aT(\lceil n/b \rceil)+O(n^{d}) \texttt{ for } a>0,b>1 \texttt{, and } d\geq 0,\]
then,
\[T(n)=\left\{\begin{array}{lr}
            O(n^{d}) & if d>log_{b}a\\
            O(n^{d}logn) & if d=log_{b}a\\
            O(n^{log_{b}a}) & if d < lob_{b}a
        \end{array}
        \right.\]

\subsection*{Fast Fourier Transform}
Todo Add FFT Information Here

\subsection*{Depth First Search}
\begin{verbatim}
def explore(G,v): #Where G = (V,E) of a Graph
    visited(v) = true
    previsit(v)
    for each edge(v,u) in E:
        if not visited(u):
            explore(u)
    postvisit(v)

def dfs(G):
    for all v in V:
        if not visited(v):
            explore(v)
\end{verbatim}
Previsit = count till node added to the queue\\
Postvisit = count till you leave the given node\\
A directed Graph has a cycle if it has a back edge found during DFS

\subsection*{Directed Acyclic Graphs}
Every DAG has a source and sink\\
Todo add more properties

\subsection*{Greedy Algorithms}
\subsubsection*{Kruskal's MST Algorithm}
Repeatedly add the next lightest edge that doesn't produce a cycle.

\subsubsection*{Properties of Trees (undirected acyclic graphs)}
\begin{itemize}
    \item A tree with n nodes has n-1 edges
    \item Any connected undirected graph G(V,E), with \(|E|=|V|-1\) is a tree
    \item An undirected graph is a tree if and only if there is a unique path between any pair of nodes.
\end{itemize}

\subsubsection*{Cut Property}
Suppose edges X are part of a minimum spanning tree of \(G=(V,E)\). Pick any subset of nodes S for which X does not cross between S and V-S, and let e be the lightest edge across the partition. Then \(X \cup {e}\) is part of some Minimum Spanning Tree.

\subsubsection*{Prim's Algorithm}
(an alternative to Kruskal's Algorithm and similar to Dijkstras)\\
On each iteration, the subtreedefined by x grows by one edge, the lightest between a vertex in S and a vertex outside S.

\subsubsection*{Huffman Encoding}
A means to encode data using the optimal number of bits for each character given a distribution.
\begin{verbatim}
Huffman(f):
Input: An array f{1...n] of frequencies
Output: An encoding tree with n leaves

let H be a priority queue of integers, ordered by f
for i=1 to n: insert(H,i)
    i=deletemin(H), j=deletemin(H)
    create a node numbered k with children i,j
    f[k] = f[i]+f[j]
    insert(H,k)
\end{verbatim}

\subsubsection*{Horn Formulas}
Horn Formulas are a framework expressing logical facts and deriving conclusions. A Horn Clause is a possible solution to the Formulas. Variables are represented by two kinds of clauses:
\begin{enumerate}
    \item Implications, whose left-hand side is an AND of any numbers of positive literals and whose right-hand side is a signle positive literal. ("If the conditions on the left hold, then the one on the right mush also be true.")
        \[(z \wedge w) \Rightarrow u\]
    \item Pure negative clauses, consisting of an OR of any number of negative literals.
        \[(\bar{u} \vee \bar{v} \vee \bar{y})\]
\end{enumerate}
The a greedy algorithm to solve a Horn Formula:
\begin{verbatim}
Input: a Horn formula
Output: a satisfying assignment, if one exists

set all variables to false
while there is an implication that is not satisfied:
    set the right-hand variable of the implication to true
if all pure negative clases are satisfied:
    return the assignment
return 'The formula is not satisfiable.'
\end{verbatim}

\subsubsection*{Set Cover Algorithm}
(example. This is the Schools distributed in towns problem.)\\
\begin{verbatim}
Input: A set of elements B; sets S1,...,Sm
Output: A selection of the Si whose union is B.

Repeat until all elements of B are covered:
    Pick the set Si with the largest number of uncovered elements.
\end{verbatim}

% You can even have references
\rule{0.3\linewidth}{0.25pt}
\scriptsize
\bibliographystyle{abstract}
\bibliography{refFile}
\end{multicols}
\end{document}
